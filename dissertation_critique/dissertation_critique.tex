\documentclass[twoside,11pt]{article}

% Any additional packages needed should be included after jmlr2e.
% Note that jmlr2e.sty includes epsfig, amssymb, natbib and graphicx,
% and defines many common macros, such as 'proof' and 'example'.
%
% It also sets the bibliographystyle to plainnat; for more information on
% natbib citation styles, see the natbib documentation, a copy of which
% is archived at http://www.jmlr.org/format/natbib.pdf

\usepackage{jmlr2e}

% Definitions of handy macros can go here

\newcommand{\dataset}{{\cal D}}
\newcommand{\fracpartial}[2]{\frac{\partial #1}{\partial  #2}}

% Heading arguments are {volume}{year}{pages}{date submitted}{date published}{paper id}{author-full-names}

\jmlrheading{1}{2000}{1-48}{4/00}{10/00}{meila00a}{Marina Meil\u{a} and Michael I. Jordan}

% Short headings should be running head and authors last names

\ShortHeadings{Dissertation Critique}{Meil\u{a} and Jordan}
\firstpageno{1}

\begin{document}

\title{Dissertation Critique: Exploring Machine Learning Techniques Using Patient
      Interactions In Online Health Forums to Classify Drug Safety}

\author{\name Christopher Jeschke \email cjeschke@gmail.com \\
       \addr Engineering for Professionals\\
       Johns Hopkins University\\
       Elkridge, MD 20175, USA}

\editor{n/a}

\maketitle

\begin{abstract}%   <- trailing '%' for backward compatibility of .sty file
  Patient generated health data represents an area of active research interest
  for its potential applications in monitoring the public health. The study of
  Pharmacovigilance is one such area, focused on monitoring drugs once they have been
  released to market. Dr. Brant Chee's 2011 dissertation specifically explores the
  application of machine learning techniques to messages from online forums discussing
  patient interactions with drugs in an attempt to assemble a set of machine learning
  techniques and processes to detect drugs bound for the United States Food and Drug
  Administration's watch list. Watch list drugs which are those drugs deamed to
  pose a significant safety concern for consumers. The dissertation discusses a
  progression of techniques 
\end{abstract}

\begin{keywords}
  Drug Safety, Pharmacovigilance, NLP
\end{keywords}

\section{Introduction}
Probabilistic inference has become a core technology in AI,
largely due to developments in graph-theoretic methods for the
representation and manipulation of complex probability
distributions~\citep{pearl:88}.  Whether in their guise as
directed graphs (Bayesian networks) or as undirected graphs (Markov
random fields), \emph{probabilistic graphical models} have a number
of virtues as representations of uncertainty and as inference engines.
Graphical models allow a separation between qualitative, structural
aspects of uncertain knowledge and the quantitative, parametric aspects
of uncertainty...\\

\section{Paper Criteria}
The critique should include a summary of the research reported, a discussion of the major contributions
claimed, and an assessment of the significance of those contributions and of the research itself. The
critique should also include a brief literature review of the topic related to the thesis, discussion of relevant
algorithms, and application areas for the research reported. Where appropriate, the critique should
include a comparison with other issues discussed in class. Students are encouraged to select a
dissertation that is related to their course projects.
The evaluation criteria for the critique are as follows:
• Overview of the research reported (20%)
• Review of the related literature (15%)
• Major contributions of the thesis (20%)
• Understanding of techniques and algorithms (20%)
• Application areas (15%)
• Proper construction and readability of paper (10%)

\section{Overview}




\section{Related Literature}



\section{Major Contributions}



\section{Techniques and Algorithms}



\section{Application Areas}




{\noindent \em Remainder omitted in this sample. See http://www.jmlr.org/papers/ for full paper.}

% Acknowledgements should go at the end, before appendices and references

\acks{We would like to acknowledge support for this project
from the National Science Foundation (NSF grant IIS-9988642)
and the Multidisciplinary Research Program of the Department
of Defense (MURI N00014-00-1-0637). }

% Manual newpage inserted to improve layout of sample file - not
% needed in general before appendices/bibliography.

\newpage

\appendix
\section*{Appendix A.}
\label{app:theorem}

% Note: in this sample, the section number is hard-coded in. Following
% proper LaTeX conventions, it should properly be coded as a reference:

%In this appendix we prove the following theorem from
%Section~\ref{sec:textree-generalization}:

In this appendix we prove the following theorem from
Section~6.2:

\noindent
{\bf Theorem} {\it Let $u,v,w$ be discrete variables such that $v, w$ do
not co-occur with $u$ (i.e., $u\neq0\;\Rightarrow \;v=w=0$ in a given
dataset $\dataset$). Let $N_{v0},N_{w0}$ be the number of data points for
which $v=0, w=0$ respectively, and let $I_{uv},I_{uw}$ be the
respective empirical mutual information values based on the sample
$\dataset$. Then
\[
	N_{v0} \;>\; N_{w0}\;\;\Rightarrow\;\;I_{uv} \;\leq\;I_{uw}
\]
with equality only if $u$ is identically 0.} \hfill\BlackBox

\noindent
{\bf Proof}. We use the notation:
\[
P_v(i) \;=\;\frac{N_v^i}{N},\;\;\;i \neq 0;\;\;\;
P_{v0}\;\equiv\;P_v(0)\; = \;1 - \sum_{i\neq 0}P_v(i).
\]
These values represent the (empirical) probabilities of $v$
taking value $i\neq 0$ and 0 respectively.  Entropies will be denoted
by $H$. We aim to show that $\fracpartial{I_{uv}}{P_{v0}} < 0$....\\

{\noindent \em Remainder omitted in this sample. See http://www.jmlr.org/papers/ for full paper.}


\vskip 0.2in
\bibliography{sample}

\end{document}
